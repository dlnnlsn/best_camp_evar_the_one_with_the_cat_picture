\documentclass[11pt]{article}
\begin{document}

\begin{center}
\textbf{Combinatorics Problems that could be used in tests}
\end{center}

\begin{enumerate}

\item [INT] \textit{C Elegans} is a type of worm that has exactly 302 neurons. A neuron can either be `on' or `off', and each pattern of neurons produces a different action. Assume that when a neuron is `on', its signal is passed to any neurons to which it is connected, and they are turned on too. All neurons are connected in some way, and a pair of neurons can only be connected in one direction. A cycle occurs when a path can be followed from a neuron by turning on a neuron to which it is connected, and so on, until the original neuron is turned on again. Of the patterns of connectivity among 4 neurons, how many contain cycles?

\item [INT/SEN] A snake trader has 2017 ravenous snakes with different tail lengths. The snakes are well-trained and will stay still once placed, but each will eat any snake it can see that has a shorter tail than it! Can the snakes be lined up such that no snake can see another snake with a shorter tail length? If so, how many ways are there to do this? Snakes can only see in one direction along the line, can see all the snakes in that direction, and can face either direction. 	

\item [INT/SEN] \textit{Austrian MO 2011, Final Round} \\ Each brick of a set has 5 holes in a horizontal row. We can either place pins into individual holes or brackets into two neighboring holes. No hole is allowed to remain empty. We place $n$ such bricks in a row in order to create patterns running from left to right, in which no two brackets are allowed to follow another, and no three pins may be in a row. How many such patterns of bricks can be created?
\end{enumerate}

\end{document}