\documentclass[a4paper,12pt]{article}

% Important Packages:
 \usepackage{amsmath}    % need for subequations
 \usepackage{amsfonts}
 \usepackage{amsthm}
 \usepackage{graphicx}   % need for figures
 \usepackage{verbatim}   % useful for program listings
 %\usepackage{enumerate}
 
 % Useful macros 
 \def\tcb#1{\color{blue}{#1}}
 \def\tcr#1{\color{red}{#1}}	
 \def\tcg#1{\color{green}{#1}}
 \def\be{\begin{eqnarray}}	 	\def\ee{\end{eqnarray}}
 \def\bea{\begin{eqnarray}}	 	\def\eea{\end{eqnarray}}
 \def\bean{\begin{eqnarray*}}	\def\eean{\end{eqnarray*}}
 
 \def\D{\displaystyle}
 \def\T{\textstyle}
 \def\l{\left}
 \def\r{\right}
 \def\nf{n_{\!f}} % quark flavours
 \def\pa{\partial}
 \def\eg{e.\,g.}
 \def\ie{i.\,e.}

 \def\be{\begin{equation}}
 \def\ee{\end{equation}}
 \def\bea{\begin{eqnarray}}
 \def\eea{\end{eqnarray}}
 \def\bean{\begin{eqnarray*}}
 \def\eean{\end{eqnarray*}}
 \def\gsim{\mathrel{\rlap{\lower0.2em\hbox{$\sim$}}\raise0.2em\hbox{$>$}}}
 \def\ksim{\mathrel{\rlap{\lower0.2em\hbox{$\sim$}}\raise0.2em\hbox{$<$}}}
 \def\kg{\mathrel{\rlap{\lower0.25em\hbox{$>$}}\raise0.25em\hbox{$<$}}}
 
 \def\AA{${\buildrel_{\circ} \over {\mathrm{A}}}$}
 \def\bm#1{\mbox{\boldmath$#1$}}
 \newcommand{\eq}[1]{(\ref{#1})} 
 \def\pd{\partial}
 \def\d{\textrm{d}} 
 \def\T{\textstyle}
 \def\eg{e.\,g.}	% exempli gratia (for the sake of example)
 \def\ie{i.\,e.}	% id est (that is)


 % Page configuration:
 \topmargin -2.0cm
 \oddsidemargin -0.85cm
 \evensidemargin -0.85cm
 \textwidth 18cm
 \textheight 24cm
 

\begin{document}

\begin{center}
\textbf{Stellenbosch Camp December 2017 \\ Intermediate Test 3} \\
\textbf{Solutions}
\end{center}


\begin{enumerate}
    
    % QUESTION 1
    \item[1.] 
    
    
    % QUESTION 2
    \item[2.] Let the feet of the altitudes from $X$ onto $BP$ and $AP$ be $E$a dn $F$ respectively, and let the feet of the altitudes from $Y$ onto $DP$ and $CP$ be $G$ and $H$ respectively. Then triangles $XFP$ and $YHP$ are similar since $\angle XPF = \angle YPH$ and $\angle XFP = 90^\circ = \angle YHP$, but $XP = YP$ and so they are in fact congruent; hence $HP = FP$. Also, triangles $BFP$ and $DHP$ are similar since $\angle BFP = 90^\circ = \angle DHP$ and $\angle BPF = \angle DPH$, but since $FP = HP$ they are in fact congruent and so $BP = DP$. Similarly $AC = CP$ and so the diagonals of $ABCD$ bisect each other; hence $ABCD$ is a parallelogram.
    
    
    % QUESTION 3
    \item[3.] 
    
    
    % QUESTION 4
    \item[4.]     

    % QUESTION 5
    \item[5.]  % AoPS, Inequalities Marathon, Problem 25
    

\end{enumerate}

\end{document}

