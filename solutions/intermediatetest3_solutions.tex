\documentclass[a4paper,12pt]{article}

% Important Packages:
 \usepackage{amsmath}    % need for subequations
 \usepackage{amsfonts}
 \usepackage{amsthm}
 \usepackage{graphicx}   % need for figures
 \usepackage{verbatim}   % useful for program listings
 %\usepackage{enumerate}
 
 % Useful macros 
 \def\tcb#1{\color{blue}{#1}}
 \def\tcr#1{\color{red}{#1}}	
 \def\tcg#1{\color{green}{#1}}
 \def\be{\begin{eqnarray}}	 	\def\ee{\end{eqnarray}}
 \def\bea{\begin{eqnarray}}	 	\def\eea{\end{eqnarray}}
 \def\bean{\begin{eqnarray*}}	\def\eean{\end{eqnarray*}}
 
 \def\D{\displaystyle}
 \def\T{\textstyle}
 \def\l{\left}
 \def\r{\right}
 \def\nf{n_{\!f}} % quark flavours
 \def\pa{\partial}
 \def\eg{e.\,g.}
 \def\ie{i.\,e.}

 \def\be{\begin{equation}}
 \def\ee{\end{equation}}
 \def\bea{\begin{eqnarray}}
 \def\eea{\end{eqnarray}}
 \def\bean{\begin{eqnarray*}}
 \def\eean{\end{eqnarray*}}
 \def\gsim{\mathrel{\rlap{\lower0.2em\hbox{$\sim$}}\raise0.2em\hbox{$>$}}}
 \def\ksim{\mathrel{\rlap{\lower0.2em\hbox{$\sim$}}\raise0.2em\hbox{$<$}}}
 \def\kg{\mathrel{\rlap{\lower0.25em\hbox{$>$}}\raise0.25em\hbox{$<$}}}
 
 \def\AA{${\buildrel_{\circ} \over {\mathrm{A}}}$}
 \def\bm#1{\mbox{\boldmath$#1$}}
 \newcommand{\eq}[1]{(\ref{#1})} 
 \def\pd{\partial}
 \def\d{\textrm{d}} 
 \def\T{\textstyle}
 \def\eg{e.\,g.}	% exempli gratia (for the sake of example)
 \def\ie{i.\,e.}	% id est (that is)


 % Page configuration:
 \topmargin -2.0cm
 \oddsidemargin -0.85cm
 \evensidemargin -0.85cm
 \textwidth 18cm
 \textheight 24cm
 

\begin{document}

\begin{center}
\textbf{Stellenbosch Camp December 2017 \\ Intermediate Test 3} \\
\textbf{Solutions}
\end{center}


\begin{enumerate}
    
    % QUESTION 1
    \item[1.] 
    
    
    % QUESTION 2
    \item[2.] 
    
    
    % QUESTION 3
    \item[3.] 
    
    
    % QUESTION 4
    \item[4.] Let
    \[
        n = \prod_{i=1}^{k} p_i^{a_i}
    \]
    be the prime decomposition of $n$. Then the number of divisors of $n$ is
    equal to
    \[
        (a_1 + 1)(a_2 + 1)(a_3 + 1) \cdots (a_k + 1).
    \]

    This must be equal to $9$, and so we can see that $n$ must either be of the
    form $n = p^8$ for some prime number $p$, or of the form $n = p^2 q^2$ for
    some prime numbers $p$ and $q$. It remains to show that both of these cases
    work.

    For $n = p^8$, we have the following arrangement of its factors:
    \[
        \begin{array}{|c|c|c|}
            \hline
            p^3 & p^8 & p \\
            \hline
            p^2 & p^4 & p^6 \\
            \hline
            p^7 & 1 & p^5 \\
            \hline
        \end{array}
    \]

    For $n = p^2 q^2$, we have the following arrangement of its factors:
    \[
        \begin{array}{|c|c|c|}
            \hline
            p^2 q & q^2 & p \\
            \hline
            1 & p q & p^2 q^2 \\
            \hline
            pq^2 & p^2 & q \\
            \hline
        \end{array}
    \]

    % QUESTION 5
    \item[5.]  % AoPS, Inequalities Marathon, Problem 25
    

\end{enumerate}

\end{document}

