\documentclass[a4paper,12pt]{article}

% Important Packages:
 \usepackage{amsmath}    % need for subequations
 \usepackage{amsfonts}
 \usepackage{amsthm}
 \usepackage{graphicx}   % need for figures
 \usepackage{verbatim}   % useful for program listings
 %\usepackage{enumerate}
 \usepackage[cm]{fullpage}
 \usepackage{fancyvrb}
 
 % Useful macros 
 \def\tcb#1{\color{blue}{#1}}
 \def\tcr#1{\color{red}{#1}}  
 \def\tcg#1{\color{green}{#1}}
 \def\be{\begin{eqnarray}}     \def\ee{\end{eqnarray}}
 \def\bea{\begin{eqnarray}}     \def\eea{\end{eqnarray}}
 \def\bean{\begin{eqnarray*}}  \def\eean{\end{eqnarray*}}
 
 \def\D{\displaystyle}
 \def\T{\textstyle}
 \def\l{\left}
 \def\r{\right}
 \def\nf{n_{\!f}} % quark flavours
 \def\pa{\partial}
 \def\eg{e.\,g.}
 \def\ie{i.\,e.}

 \def\be{\begin{equation}}
 \def\ee{\end{equation}}
 \def\bea{\begin{eqnarray}}
 \def\eea{\end{eqnarray}}
 \def\bean{\begin{eqnarray*}}
 \def\eean{\end{eqnarray*}}
 \def\gsim{\mathrel{\rlap{\lower0.2em\hbox{$\sim$}}\raise0.2em\hbox{$>$}}}
 \def\ksim{\mathrel{\rlap{\lower0.2em\hbox{$\sim$}}\raise0.2em\hbox{$<$}}}
 \def\kg{\mathrel{\rlap{\lower0.25em\hbox{$>$}}\raise0.25em\hbox{$<$}}}
 
 \def\AA{${\buildrel_{\circ} \over {\mathrm{A}}}$}
 \def\bm#1{\mbox{\boldmath$#1$}}
 \newcommand{\eq}[1]{(\ref{#1})} 
 \def\pd{\partial}
 \def\d{\textrm{d}} 
 \def\T{\textstyle}
 \def\eg{e.\,g.}  % exempli gratia (for the sake of example)
 \def\ie{i.\,e.}  % id est (that is)


\begin{document}

\begin{center}
\textbf{Stellenbosch Camp December 2017 \\ Intermediate Test 5} \\
\textbf{Solutions}
\end{center}


\begin{enumerate}
  
  % QUESTION 1
  \item[1.] \emph{Find all pairs $(m,n)$ of positive integers which satisfy the equation
  \[ mn^2 = 100(n+1). \]} 
  
  Note that $n$ and $n+1$ differ by 1, and so are coprime. Therefore $n^2$ must divide 100. Whichever value $n$ takes, $m$ can be found as $\frac{(n+1)\cdot100}{n^2}$. So $(m,n)$ can take only the following solutions: $(200,1)$; $(75,2)$; $(24,5)$; $(11,10)$.
  
  % QUESTION 2
  \item[2.] \emph{Point $M$ and $N$ are chosen on the sides $BC$ and $CD$ of a square $ABCD$, respectively, so that $\angle MAN = 45^\circ$. Points $P$ and $Q$ are the intersections of the diagonal $BD$ with $AM$ and $AN$, respectively. Prove that $P$ and $Q$ lie on the circle with diameter $MN$.}
  
  $\angle PAN=\angle PDN = 45^\circ$ (diagonal $BD$), so $APND$ is a cyclic quadrilateral, since the angles subtended by chord $PN$ are equal. Thus $\angle ANP= \angle PDA = 45^\circ$ ($AP$ common chord), and $\angle APN=90^\circ$ ($\triangle APN$). Then $\angle MPN=90^\circ$ (straight line), so $P$ is subtended by diameter $MN$ in circle $MNP$. By symmetry, the same must apply for point $Q$, using the cyclic quadrilateral $BMQA$. Therefore $P$ and $Q$ both lie on the circle with diameter $MN$.
  
  % QUESTION 3
  \item[3.] 


  % QUESTION 4
  \item[4.] \emph{Find all functions $f: \mathbb{R} \to \mathbb{R}$ such that for all $x,y \in \mathbb{R}$,
\[ f(x+f(f(y))) = y+f(f(x)). \]}

  \textbf{Solution 1:} First, let us prove that $f$ is injective. Assume $f(x)=f(y)$. Applying $f$ on either side, adding $x$ and then applying $f$ again gives: $f(x+f(f(x))) = f(x+f(f(y)))$. Applying the condition in the problem statement to both sides then gives:  
  \begin{equation*}
    x+f(f(x)) = f(x+f(f(x))) = f(x+f(f(y))) = y+f(f(x)) 
  \end{equation*} 
  and so $x+f(f(x)) = y+f(f(x))$ which implies $x=y$, and so we have $f$ injective.
  
  Now letting $x=y=0$, we have: $f(f(f(0))) = f(f(0))$ which from injectivity (i.e. we can cancel $f$'s), implies that $f(0)=0$. Letting $y=0$, we get: $f(x+f(f(0))) = f(f(x))$. Using that $f(0)=0$, this simplifies to $f(x)=f(f(x))$. Finally, from injectivity, we get $x=f(x)$. All that remains is to check this is indeed a solution. Subbing $f(x)=x$ into the original equation gives:
  \begin{equation*}
    f(x+f(y)) = f(x+y) = x+y = y+x = y+f(x) = y+f(f(x))
  \end{equation*}
  and so $f(x)=x$ is a solution.
  
  \textbf{Solution 2:}
  From the problem statement we have: 
  \begin{equation}\label{given}
    f(x+f(f(y)) = y+f(f(x))
  \end{equation} 
  Swapping $x$ and $y$ around, we get: 
  \begin{equation}\label{swap}
    f(y+f(f(x)) = x+f(f(y))
  \end{equation} 
  Noticing that the right-hand side of \eqref{swap} is the same as the left-hand side argument in \eqref{given}, we take $f$'s on both sides and apply \eqref{given}:
  \begin{equation*}
    f(f(y+f(f(x)))) = f(x+f(f(y)) = y+f(f(x))
  \end{equation*}
  Since $z = y+f(f(x))$ may take on any $\mathbb{R}$ value, the above implies that $f(f(z)) = z$, $\forall z\in \mathbb{R}$. Then, taking $x=0$ in \eqref{given} and using $f(f(z))=z$, we get:
  \begin{equation*}
    f(y) = f(f(f(y))) = f(0+f(f(y)) = y+f(f(0)) = y +0 = y
  \end{equation*} 
  and so once again, $f(y)=y$ is a candidate for a solution. The check follows as in the previous solution.
  

  % QUESTION 5
  \item[5.] \emph{Find all primes $p$ such that $\displaystyle \frac{2^{p-1}-1}{p}$ is the square of an integer.}
  
  In order for $\frac{2^{p-1}-1}{p}$ to be a square, we must have that $2^{p-1}-1 = p a^2$ for some $a\in \mathbb{N}$. In fact, since we can factorize this as a difference of squares:  $2^{p-1}-1 = (2^\frac{p-1}{2}+1)(2^\frac{p-1}{2}-1)$, we must have that one bracket is a square and the other is a square times $p$. For $k,q \in \mathbb{N}$, we have two cases:
  
  \underline{\textbf{Case 1:}} $2^\frac{p-1}{2}+1=k^2$, $2^\frac{p-1}{2}-1 = pq^2$

  We can write $2^\frac{p-1}{2} = k^2-1 = (k-1)(k+1)$. Both $k-1$ and $k+1$ must then be powers of 2 which differ by 2. The only such numbers are 2 and 4. This sets $k=3$ and further implies that $\frac{p-1}{2}=3$ and so $p=7$. Check: $2^{7-1}-1 = 64-1 = 7\cdot 3^2$ and so $\frac{2^{7-1}-1}{7}$ is indeed a square.
  
  \underline{\textbf{Case 2:}} $2^\frac{p-1}{2}-1=k^2$, $2^\frac{p-1}{2}+1 = pq^2$
  
  Let's look at $2^\frac{p-1}{2}-1=k^2$ modulo 4. All powers of 2 are 0 $\pmod{4}$ apart from $2^0=1 \equiv 1$ and $2^1=2 \equiv 2 \pmod{4}$. Since, squares may only be 0 or 1 $\pmod{4}$, we must have $\frac{p-1}{2} = 0$ or 1 in order for $2^\frac{p-1}{2}-1 \equiv 0$ or $1 \pmod{4}$ respectively. We cannot have $\frac{p-1}{2} = 0$ since then $p=1$ which is not prime. However, $\frac{p-1}{2} = 1$ gives $p=3$ which is prime. Check: $2^{3-1}-1 = 2^2-1 = 3\cdot 1^2$ and so $\frac{2^{3-1}-1}{3}$ is indeed a square. 
  
  Therefore, there are two solutions, namely $p=3$ and $p=7$.
  
  
  % QUESTION 6
  \item[6.] \emph{Let $ABCD$ be a convex quadrilateral with no parallel sides. Make a parallelogram on each two consecutive sides. Show that among these 4 new points, there is only one point inside quadrilateral $ABCD$.}
  
  Let $\angle A, \angle B, \angle C$ and $\angle D$ denote the interior angles of quadrilateral $ABCD$ at vertices $A$, $B$, $C$ and $D$ respectively, and let $P_A$ denote the vertex opposite $A$ in the parallelogram formed by $D$, $A$ and $B$, and similarly for $P_B$, $P_C$ and $P_D$.
  
  Now for $P_A$ to lie inside the quadrilateral $ABCD$, we must have that $\angle ABP_A < \angle ABC = \angle B$ and $\angle ADP_A < \angle ADC = \angle D$. However, since $ABP_AD$ is a parallelogram, $\angle ABP_A = \angle ADP_A = 180^\circ - \angle A$ and so $P_A$ lies inside the quadrilateral if and only if $\angle A + \angle B > 180^\circ$ and $\angle A + \angle D > 180^\circ$, and similarly for $P_B$, $P_C$ and $P_D$.
  
  However, $\angle A + \angle B + \angle C + \angle D = 360^\circ$ and (since no two sides of $ABCD$ are parallel) none of $\angle A + \angle B$, $\angle B + \angle C$, $\angle C + \angle D$ or $\angle D + \angle A$ is equal to $180^\circ$. Moreover, $\angle A + \angle B$ is less (greater) than $180^\circ$ exactly as $\angle C + \angle D$ is greater (less) than $180^\circ$ and $\angle B + \angle C$ is less (greater) than $180^\circ$ exactly as $\angle D + \angle A$ is greater (less) than $180^\circ$. Hence it follows that for exactly one of $A$, $B$, $C$ and $D$ (let's call it $X$) $P_X$ lies inside $ABCD$.
  
  
\end{enumerate}


\centering
\begin{BVerbatim}
\end{BVerbatim}

\end{document}

