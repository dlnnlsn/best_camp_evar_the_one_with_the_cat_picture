\documentclass[a4paper,12pt]{article}

% Important Packages:
 \usepackage{amsmath}    % need for subequations
 \usepackage{amsfonts}
 \usepackage{amsthm}
 \usepackage{graphicx}   % need for figures
 \usepackage{verbatim}   % useful for program listings
 %\usepackage{enumerate}
 \usepackage[cm]{fullpage}
 \usepackage{fancyvrb}
 
 % Useful macros 
 \def\tcb#1{\color{blue}{#1}}
 \def\tcr#1{\color{red}{#1}}	
 \def\tcg#1{\color{green}{#1}}
 \def\be{\begin{eqnarray}}	 	\def\ee{\end{eqnarray}}
 \def\bea{\begin{eqnarray}}	 	\def\eea{\end{eqnarray}}
 \def\bean{\begin{eqnarray*}}	\def\eean{\end{eqnarray*}}
 
 \def\D{\displaystyle}
 \def\T{\textstyle}
 \def\l{\left}
 \def\r{\right}
 \def\nf{n_{\!f}} % quark flavours
 \def\pa{\partial}
 \def\eg{e.\,g.}
 \def\ie{i.\,e.}

 \def\be{\begin{equation}}
 \def\ee{\end{equation}}
 \def\bea{\begin{eqnarray}}
 \def\eea{\end{eqnarray}}
 \def\bean{\begin{eqnarray*}}
 \def\eean{\end{eqnarray*}}
 \def\gsim{\mathrel{\rlap{\lower0.2em\hbox{$\sim$}}\raise0.2em\hbox{$>$}}}
 \def\ksim{\mathrel{\rlap{\lower0.2em\hbox{$\sim$}}\raise0.2em\hbox{$<$}}}
 \def\kg{\mathrel{\rlap{\lower0.25em\hbox{$>$}}\raise0.25em\hbox{$<$}}}
 
 \def\AA{${\buildrel_{\circ} \over {\mathrm{A}}}$}
 \def\bm#1{\mbox{\boldmath$#1$}}
 \newcommand{\eq}[1]{(\ref{#1})} 
 \def\pd{\partial}
 \def\d{\textrm{d}} 
 \def\T{\textstyle}
 \def\eg{e.\,g.}	% exempli gratia (for the sake of example)
 \def\ie{i.\,e.}	% id est (that is)


\begin{document}

\begin{center}
\textbf{Stellenbosch Camp December 2017 \\ Intermediate Test 5} \\
\textbf{Solutions}
\end{center}


\begin{enumerate}
    
    % QUESTION 1
    \item[1.] 
    
    
    % QUESTION 2
    \item[2.] 
    
    
    % QUESTION 3
    \item[3.] 


    % QUESTION 4
    \item[3.] 
    

    % QUESTION 5
    \item[5.] 
    
    
    % QUESTION 6
    \item[6.] \emph{Let $ABCD$ be a convex quadrilateral with no parallel sides. Make a parallelogram on each two consecutive sides. Show that among these 4 new points, there is only one point inside quadrilateral $ABCD$.}
    
    Let $\angle A, \angle B, \angle C$ and $\angle D$ denote the interior angles of quadrilateral $ABCD$ at vertices $A$, $B$, $C$ and $D$ respectively, and let $P_A$ denote the vertex opposite $A$ in the parallelogram formed by $D$, $A$ and $B$, and similarly for $P_B$, $P_C$ and $P_D$.
    
    Now for $P_A$ to lie inside the quadrilateral $ABCD$, we must have that $\angle ABP_A < \angle ABC = \angle B$ and $\angle ADP_A < \angle ADC = \angle D$. However, since $ABP_AD$ is a parallelogram, $\angle ABP_A = \angle ADP_A = 180^\circ - \angle A$ and so $P_A$ lies inside the quadrilateral if and only if $\angle A + \angle B > 180^\circ$ and $\angle A + \angle D > 180^\circ$, and similarly for $P_B$, $P_C$ and $P_D$.
    
    However, $\angle A + \angle B + \angle C + \angle D = 360^\circ$ and (since no two sides of $ABCD$ are parallel) none of $\angle A + \angle B$, $\angle B + \angle C$, $\angle C + \angle D$ or $\angle D + \angle A$ is equal to $180^\circ$. Moreover, $\angle A + \angle B$ is less (greater) than $180^\circ$ exactly as $\angle C + \angle D$ is greater (less) than $180^\circ$ and $\angle B + \angle C$ is less (greater) than $180^\circ$ exactly as $\angle D + \angle A$ is greater (less) than $180^\circ$. Hence it follows that for exactly one of $A$, $B$, $C$ and $D$ (let's call it $X$) $P_X$ lies inside $ABCD$.
    
    
\end{enumerate}


\centering
\begin{BVerbatim}
\end{BVerbatim}

\end{document}

