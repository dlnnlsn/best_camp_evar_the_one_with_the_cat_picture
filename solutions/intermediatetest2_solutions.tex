\documentclass[12pt]{article}
\usepackage{amsmath,amsfonts}
\usepackage{fullpage}

\title{Intermediate Test 2 Solutions}
\author{Stellenbosch Camp 2017}

\begin{document} \maketitle

\begin{enumerate}

\item[1.] \emph{Prove that there are an infinite number of solutions to $x^3 + y^3 = z^2$ over the positive integers.}

For any natural number $a$, we have that
\[
    (2a^2)^3 + (2a^2)^3 = (4a^3)^2.
\]

\item[2.] \emph{Determine all triples $(x,y,z)$ of positive real numbers that satisfy
\begin{align*}
	3\lfloor{x}\rfloor -\{y\} +\{z\} &= 20.3 \\
	3\lfloor{y}\rfloor +5\lfloor{z}\rfloor -\{x\} &= 15.1 \\
	\{y\} +\{z\} &= 0.9.
\end{align*}
}

\item[3.] \emph{Let $ABC$ be an acute-angled triangle with $\angle BAC = 75^\circ$. Let $P$ the midpoint of the side $BC$, and let $M$ and $N$ be the feet of the altitudes from vertices $B$ and $C$ respectively. Determine the angle $\angle MPN$.}

Since $\angle BAC = 75^\circ$, we have that $\angle MAB = 15^\circ$. Note that
$MNBC$ is a cyclic quad with center $P$. The chord $MN$ subtends an angle of
$15^\circ$ at $B$, and hence subtends an angle of $2 \times 15^\circ = 30^\circ$
at $P$.

\item[4.] \emph{A snake trader has 2017 ravenous snakes with different tail lengths which he wants to place in a line from west to east, with each snake facing either east or west. The snakes are well-trained and will stay still once placed, but each will eat any snake it can see that has a shorter tail than it! (Each snake can see all the snakes in the direction in which it is facing.) Can the snakes be lined up such that no snake can see another snake with a shorter tail length? If so, how many ways are there to do this?}

Clearly the snakes can always be lined up so that no snake can see another snake with a shorter tail length, since for example the snake trader can always line them up facing west in descending order. The number of suitable arrangements for $n$ snakes is $2^n$, which we prove using induction.

\textbf{Base case, $n =1$:} The snake can either face west or east, and hence can be lined up in 2 ways.

\textbf{Inductive hypothesis, $n=k$:} Assume that for $n=k$ snakes, they can be lined up in $2^k$ ways.

\textbf{Inductive step, $n=k+1$:} Since all snakes have distinct tail lengths, we can find and remove the tallest snake from the $k+1$ snakes. The other $k$ snakes can be lined up in $2^k$ ways, using our inductive hypothesis. The remaining snake can only be inserted on the far sides of any line of the $k$ snakes, since if it is inserted between any two snakes it must be taller than both of them, and hence no matter which way it faces it would have a snake to eat. Similarly, on either the west or the east edge, the snake must face away from the other snakes, else again it would see a shorter snake. So there are 2 ways to add the last snake, and thus the total arrangements of snakes numbers $2^{k+1}$.

Thus by induction, $n$ snakes can be lined up in $2^n$ ways, so for 2017 snakes there are $2^{2017}$ arrangements.

\item[5.] \emph{Find all functions $f : \mathbb{R} \to \mathbb{R}$ such that
	\[f(x^2-y^2) = x^2 - f(y^2)\]
for all $x,y \in \mathbb{R}$.}



\end{enumerate}

\end{document}

