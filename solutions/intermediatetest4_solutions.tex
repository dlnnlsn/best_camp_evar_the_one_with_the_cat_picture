\documentclass[a4paper,12pt]{article}

% Important Packages:
 \usepackage{amsmath}    % need for subequations
 \usepackage{amsfonts}
 \usepackage{amsthm}
 \usepackage{graphicx}   % need for figures
 \usepackage{verbatim}   % useful for program listings
 %\usepackage{enumerate}
 \usepackage[cm]{fullpage}
 
 % Useful macros 
 \def\tcb#1{\color{blue}{#1}}
 \def\tcr#1{\color{red}{#1}}	
 \def\tcg#1{\color{green}{#1}}
 \def\be{\begin{eqnarray}}	 	\def\ee{\end{eqnarray}}
 \def\bea{\begin{eqnarray}}	 	\def\eea{\end{eqnarray}}
 \def\bean{\begin{eqnarray*}}	\def\eean{\end{eqnarray*}}
 
 \def\D{\displaystyle}
 \def\T{\textstyle}
 \def\l{\left}
 \def\r{\right}
 \def\nf{n_{\!f}} % quark flavours
 \def\pa{\partial}
 \def\eg{e.\,g.}
 \def\ie{i.\,e.}

 \def\be{\begin{equation}}
 \def\ee{\end{equation}}
 \def\bea{\begin{eqnarray}}
 \def\eea{\end{eqnarray}}
 \def\bean{\begin{eqnarray*}}
 \def\eean{\end{eqnarray*}}
 \def\gsim{\mathrel{\rlap{\lower0.2em\hbox{$\sim$}}\raise0.2em\hbox{$>$}}}
 \def\ksim{\mathrel{\rlap{\lower0.2em\hbox{$\sim$}}\raise0.2em\hbox{$<$}}}
 \def\kg{\mathrel{\rlap{\lower0.25em\hbox{$>$}}\raise0.25em\hbox{$<$}}}
 
 \def\AA{${\buildrel_{\circ} \over {\mathrm{A}}}$}
 \def\bm#1{\mbox{\boldmath$#1$}}
 \newcommand{\eq}[1]{(\ref{#1})} 
 \def\pd{\partial}
 \def\d{\textrm{d}} 
 \def\T{\textstyle}
 \def\eg{e.\,g.}	% exempli gratia (for the sake of example)
 \def\ie{i.\,e.}	% id est (that is)


\begin{document}

\begin{center}
\textbf{Stellenbosch Camp December 2017 \\ Intermediate Test 4} \\
\textbf{Solutions}
\end{center}


\begin{enumerate}
    
    % QUESTION 1
    \item[1.] Since $72 = 8 \cdot 9$, the number $\overline{a2017b}$ must satisfy divisibility rules for both 8 and 9. For 8, we must have that the number formed by the last 3 digits, $\overline{17b}$, is a multiple of 8. The only multiple of 8 between 170 and 179 is 176, fixing $b=6$. Then for 9, the sum of the digits must be a multiple of 9. Since $2+0+1+7 = 10$, we must have $a+b=8$ or 17 to bring the total up to 18 or 27 respectively. $17-6 = 11$ which is greater than 9 and so we must have that $a=8-6=2$.
    
    
    % QUESTION 2
    \item[2.] For a \textit{catty} number with $n$-digits, there are two options for each digit (each digit may be either a 2 or a 3). Therefore, the number of $n$-digit catty numbers is simply $2^n$. The total number of catty numbers with $k\leq n$ digits is then $S_n = \sum_{k=1}^{n}2^k = 2^1+2^2+\cdots 2^n$. To find a more compact expression for $S_n$, let us multiply $S_n$ by 2 to get $2^2+2^3+\cdots 2^{n+1}$. Subtracting $S_n$ from $2S_n$ then gives $S_n = (2^2+2^3+\cdots 2^{n+1}) - (2^1+2^2+\cdots 2^n) = 2^{n+1}-2$. $S_{10} = 2^{10+1}-2 =2046$ and so the $2047^{\text{th}}$ is the first \textit{catty} number with 11 digits i.e. 22222222222. The $2048^{\text{th}}$ and $2049^{\text{th}}$ \textit{catty} numbers are then 22222222223 and 22222222232 respectively, leaving the $2050^{\text{th}}$ as 22222222233.
    
    % QUESTION 3
    \item[3.] Applying Cauchy in Engel-form to the left-hand side, we have for $x,y,z \in \mathbb{R}^+ \cup \{0\}$:
\begin{equation*}
\frac{1}{1+x}+\frac{1}{1+y}+\frac{1}{1+z} \geq \frac{(1+1+1)^2}{(1+x)+(1+y)+(1+z)} = \frac{3^2}{3+(x+y+z)}
\end{equation*}
Using that $x+y+z\leq 3$, we have further that:
\begin{equation*}
\frac{3^2}{3+(x+y+z)}\geq \frac{3^2}{3+3} = \frac{3}{2}
\end{equation*}
and so, as desired, we have:
\begin{equation*}
\frac{1}{1+x}+\frac{1}{1+y}+\frac{1}{1+z} \geq \frac{3}{2}
\end{equation*}


    % QUESTION 4
    \item[3.] Let $O$ be the centre of the circle $\Gamma$, and let the angle
        bisector of $\angle AXC$ meet $\Gamma$ at the points $S$ on the arc
        $AC$, and the point $T$ on the arc $BD$.  Since $M$ and $N$
        are the midpoints of the chords $AB$ and $CD$ respectively, we have that
        $OM \perp AB$ and $ON \perp CD$. Thus $MXNO$ is a cyclic quadrilateral.
        We then have that $\angle MOX = \angle MNX$ (subtended by $MX$) $=
        \angle TXD$ (since $TX \parallel MN$) $= \angle MXT$ (angle bisector) $=
        \angle XMN$ (since $TX \parallel MN$) $= \angle XON$ (subtended by
        $XN$). Thus in triangles $\triangle MOX$ and $\triangle NOX$, we have that $\angle MOX =
        \angle XON$, $\angle XMO = \angle ONX = 90^\circ$, and $OX$ is common,
        and so $\triangle MOX \equiv \triangle NOX$, giving us that $OM = ON$.
        Thus in triangles $\triangle OBM$ and $\triangle OCN$, we have $\angle
        OMB = \angle ONC = 90^\circ$, $OM = ON$, and $OB = OC$ (radii). Thus
        $\triangle OBM \equiv \triangle OCM$, and so $BM = CN$, giving us that
        $AB = 2MB = 2NC = CD$.

    % QUESTION 5
    \item[5.] Since we want $2^{m^2}-4$ to be a multiple of 7, we must have $2^{m^2} \equiv 4 \pmod{7}$. To this end, let's look at powers of 2$\pmod{7}$. Noticing that $2^3=8\equiv 1 \pmod{7}$ , we have that: $2^{3k} = (2^3)^k \equiv (1)^k = 1 \pmod{7}$, $2^{3k+1} = 2(2^{3k}) \equiv 2(1) = 2 \pmod{7}$, and $2^{3k+2} = 2^2(2^{3k}) \equiv 4(1) = 4 \pmod{7}$. We must therefore have that $m^2 = 3k+2$ for some $k\in \mathbb{N}_0$. However, $(3k)^2\equiv 0 \pmod{3}$, $(3k+1)^2\equiv 1 \pmod{3}$ and $(3k+2)^2 \equiv 2^2 \equiv 1 \pmod{3}$ and so squares can only ever be congruent to 0 or 1 (and not 2) modulo 3. Therefore, there does not exist an $m\in \mathbb{N}_0$ such that $7|(2^{m^2}-4)$.
    
    
\end{enumerate}

\end{document}

