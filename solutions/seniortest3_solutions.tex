\documentclass[a4paper, 12pt]{article}
%\documentclass{book}

% Important Packages:
 \usepackage{amsmath}    % need for subequations
 \usepackage{amsfonts}
 \usepackage{amsthm}
 \usepackage{graphicx}   % need for figures
 \usepackage{verbatim}   % useful for program listings
 %\usepackage{subfig}  % use for side-by-side figures
 %\usepackage{wrapfig}
 %\usepackage{listings}	 % creates code blocks
 %\usepackage[colorlinks=true]{hyperref}   % use for hypertext links, including
                     % those to external documents and URLs
 %\usepackage{multirow}
 %\usepackage{tikz}
 %\usepackage{enumerate}
 %\usetikzlibrary{decorations.pathreplacing,decorations.pathmorphing}
 %\usetikzlibrary{calc}
 %\usepackage[colorinlistoftodos]{todonotes}
 
 % Useful macros 
 \def\tcb#1{\color{blue}{#1}}
 \def\tcr#1{\color{red}{#1}}	
 \def\tcg#1{\color{green}{#1}}
 \def\be{\begin{eqnarray}}	 	\def\ee{\end{eqnarray}}
 \def\bea{\begin{eqnarray}}	 	\def\eea{\end{eqnarray}}
 \def\bean{\begin{eqnarray*}}	\def\eean{\end{eqnarray*}}
 
 \def\D{\displaystyle}
 \def\T{\textstyle}
 \def\l{\left}
 \def\r{\right}
 \def\nf{n_{\!f}} % quark flavours
 \def\pa{\partial}
 \def\eg{e.\,g.}
 \def\ie{i.\,e.}

 \def\be{\begin{equation}}
 \def\ee{\end{equation}}
 \def\bea{\begin{eqnarray}}
 \def\eea{\end{eqnarray}}
 \def\bean{\begin{eqnarray*}}
 \def\eean{\end{eqnarray*}}
 \def\gsim{\mathrel{\rlap{\lower0.2em\hbox{$\sim$}}\raise0.2em\hbox{$>$}}}
 \def\ksim{\mathrel{\rlap{\lower0.2em\hbox{$\sim$}}\raise0.2em\hbox{$<$}}}
 \def\kg{\mathrel{\rlap{\lower0.25em\hbox{$>$}}\raise0.25em\hbox{$<$}}}
 
 \def\AA{${\buildrel_{\circ} \over {\mathrm{A}}}$}
 \def\bm#1{\mbox{\boldmath$#1$}}
 \newcommand{\eq}[1]{(\ref{#1})} 
 \def\pd{\partial}
 \def\d{\textrm{d}} 
 \def\T{\textstyle}
 \def\eg{e.\,g.}	% exempli gratia (for the sake of example)
 \def\ie{i.\,e.}	% id est (that is)


 % Page configuration:
 \topmargin -2.0cm
 \oddsidemargin -0.85cm
 \evensidemargin -0.85cm
 \textwidth 18cm
 \textheight 24cm
 
\begin{document}
\begin{center}
\textbf{Stellenbosch Camp December 2017 \\ Senior Test 3} \\
\textbf{Solutions}
\end{center}

\begin{enumerate}
    % EGMO 2013 solutions: https://www.egmo.org/egmos/egmo2/solutions.pdf

    % QUESTION 1
    \item[1.] Let
    \[
        n = \prod_{i=1}^{k} p_i^{a_i}
    \]
    be the prime decomposition of $n$. Then the number of divisors of $n$ is
    equal to
    \[
        (a_1 + 1)(a_2 + 1)(a_3 + 1) \cdots (a_k + 1).
    \]

    This must be equal to $9$, and so we can see that $n$ must either be of the
    form $n = p^8$ for some prime number $p$, or of the form $n = p^2 q^2$ for
    some prime numbers $p$ and $q$. It remains to show that both of these cases
    work.

    For $n = p^8$, we have the following arrangement of its factors:
    \[
        \begin{array}{|c|c|c|}
            \hline
            p^3 & p^8 & p \\
            \hline
            p^2 & p^4 & p^6 \\
            \hline
            p^7 & 1 & p^5 \\
            \hline
        \end{array}
    \]

    For $n = p^2 q^2$, we have the following arrangement of its factors:
    \[
        \begin{array}{|c|c|c|}
            \hline
            p^2 q & q^2 & p \\
            \hline
            1 & p q & p^2 q^2 \\
            \hline
            pq^2 & p^2 & q \\
            \hline
        \end{array}
    \]
    
    % QUESTION 2
    \item[2.] Let's put segment $MY = MX$ from point $M$ on ray $XM$. Then $\triangle AMY = \triangle MXN$ by an angle and two adjacent edges, now we have:
    \begin{align*}
        P_{ANX} &= AX + XN + AN = AX + AY + MD > XY + MD = 2XM + MD \\
        &> XM + XD + MD = P_{MXD}
    \end{align*}
    as the projection of segment $XD$ into line $AD$ is smaller than the projection of segment $XM$, then the segment itself is smaller by length.
    
        \begin{figure}[h!]
        \centering
        \includegraphics[width=0.4\textwidth]{seniortest3_q3.PNG}
    \end{figure}
    
    
    % QUESTION 3
    \item[3.] 
    
    
    % QUESTION 4
    \item[4.] Note that if $p=2$ then the given fraction yields 4, which is indeed a perfect square. We now assume $p > 2$. By Bertrand's postulate, there exists a prime $q$, such that $\frac{p-1}{2} < q \leq p-1$. We now consider writing the denominator as a single fraction, where:
    %$$ 1+\frac{1}{2}+\dots+\frac{1}{p-1} = \frac{M}{\textrm{lcm}(1, 2, \dots, p-1)} $$
    \begin{equation} \label{eqn41}
        1+\frac{1}{2}+\dots+\frac{1}{p-1} = \frac{M}{1 \cdot 2 \cdot 3 \cdot \dots (p-1)}
    \end{equation}
    where
    $$ M = \sum_{i=1}^{p-1}  \prod_{k=1, k \neq i}^{p-1} k $$
    Noting that $q$ divides all terms in $M$ except for one, we have that $q \not | M$. Also $q$ divides the denominator of (\ref{eqn41}) exactly once. Hence, the exponent of $q$ in the original fraction is 1, which proves it cannot be a perfect square of a \textit{rational} number if $p>2$.
    
    Hence, only $p=2$ satisfies the desired property.
    

    % QUESTION 5
    \item[5.]  % AoPS, Inequalities Marathon, Problem 25
    \emph{Solution 1:} Note that we have $a^2 + b^2 + \sqrt{c} \geq 2ab + \sqrt{c} = \frac{2}{c} + \sqrt{c}$. Hence, we have
    $$ \sum_\textrm{cyc} \frac {ab}{a^2 + b^2 + \sqrt {c}} \leq \sum_\textrm{cyc} \frac{\frac{1}{c}}{\frac{2}{c} + \sqrt{c}} = \sum_\textrm{cyc} \frac{1}{2 + c\sqrt{c}} $$
    Reducing to a common denominator, we have that:
    \begin{align}
        \sum_\textrm{cyc} \frac{1}{2 + c\sqrt{c}} &= \frac{(2 + b \sqrt{b})(2 + c \sqrt{c}) + (2 + c \sqrt{c})(2 + a \sqrt{a}) + (2 + a \sqrt{a})(2 + b \sqrt{b})}{(2 + a \sqrt{a})(2 + b \sqrt{b})(2 + c \sqrt{c})} \nonumber \\
        &= \frac{4 \cdot 3 + (ab\sqrt{ab} + bc\sqrt{bc} + ca\sqrt{ca}) + 4(a\sqrt{a} + b\sqrt{b} + c\sqrt{c})}{1 + 2(ab\sqrt{ab} + bc\sqrt{bc} + ca\sqrt{ca}) + 4(a\sqrt{a} + b\sqrt{b} + c\sqrt{c}) + 8} \label{eqn51}
    \end{align} 
    
    It now remains to prove that (\ref{eqn51}) $ \leq 1$, where we now use AM-GM:
    $$  ab\sqrt {ab} + bc\sqrt {bc} + ca\sqrt {ca} \geq 3 \sqrt [3]{a^2b^2c^2\sqrt {a^2b^2c^2}} = 3 $$
    which proves that:
    $$ \sum_\textrm{cyc} \frac{1}{2 + c\sqrt{c}} \leq 1 $$
    This completes the proof.
    
	  \emph{Solution 2:} As before, we begin with $a^2 + b^2 + \sqrt{c} \geq 2ab + \sqrt{c} = 2ab + \sqrt{abc^2} = \sqrt{ab}\left(2\sqrt{ab}+c\right)$. Hence, we have
	    \[ \sum_\textrm{cyc} \frac {ab}{a^2 + b^2 + \sqrt {c}} \leq \sum_\textrm{cyc} \frac{\sqrt{ab}}{2\sqrt{ab}+c} = \frac{1}{2} - \frac{c}{4\sqrt{ab}+2c} \]
	  and so it remains to show that
	    \[ \sum_\textrm{cyc} \frac{c}{2\sqrt{ab}+c} \geq 1. \]
	  But by Cauchy's Inequality in Engel form, the left-hand side is
	    \[ \sum_\textrm{cyc} \frac{(\sqrt{c})^2}{2\sqrt{ab}+c} \geq \frac{\left(\sqrt{a}+\sqrt{b}+\sqrt{c}\right)^2}{a+b+c+2\sqrt{ab}+2\sqrt{bc}+2\sqrt{ca}} = 1, \]
	  and so the inequality is proved.
\end{enumerate}
\end{document}





