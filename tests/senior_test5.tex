\documentclass[12pt]{article}
\usepackage{amsmath,amsfonts}
\usepackage[cm]{fullpage}
\usepackage{fancyvrb}

\title{Senior Test 5}
\author{Stellenbosch Camp 2017}
\date{Time: 4 hours}


\begin{document} \maketitle

\begin{enumerate}

\item[1.] % AoPS
Find all primes $p$ such that $\displaystyle \frac{2^{p-1}-1}{p}$ is the square of an integer.


\item[2.] % Nathalie Shwarz
Consider an 8-by-8 chessboard with 17 pieces placed on it in separate squares. Prove that there are three of these pieces which lie in three different rows and three different columns.


\item[3.] % Jon Brown
Let $ABC$ be a right-angled isosceles triangle with $\angle BAC = 90^\circ$ and $AB = AC$. Let $I$ be the incentre of $\triangle ABC$ and $P$ be the intersection of $BI$ and $AC$. Let $C'$ be the midpoint of $AB$ and let $X$ be the intersection of $IC'$ and $AC$. Prove that $X$ is the midpoint of $CP$.


\item[4.] % Austrian MO 2011, Final Round
Each brick of a set has 5 holes in a horizontal row. We can either place pins into individual holes or brackets into two neighboring holes. No hole is allowed to remain empty. We place n such bricks in a row in order to create patterns running from left to right, in which no two brackets are allowed to follow another, no three pins may be in a row, and no bracket can cross over between consecutive bricks. How many such patterns of bricks can be created?


\item[5.] % Stephanie Brown
Let $f : \mathbb{N} \to \mathbb{N}$ be such that
  \[ f(f(f(z))) f(wx f(y f(z))) = z^2 f(xf(y)) f(w) .\]
for all $w, x, y, z \in \mathbb{N}$. Prove that $f(n!) \geq n!$ for all $n \in \mathbb{N}$.


\item[6.] % Number Theory Marathon, Prob 126
The number $6$ is written on the blackboard. At the $n^\mathrm{th}$ step, the integer $k$ on the board is replaced by $k + \textrm{gcd}(n, k)$. Prove that at each step, the number on the blackboard increases either by $1$ or by a prime number.

\end{enumerate}

\vfill

\centering
\begin{BVerbatim}
           _ 
          ((
           \\
            ))       
           //.--.     |\_/|
          |      `'..' a a(
           \  \      \ =_Y/=
           /   |   /  /`"`
           > /` --< <<
           \__))   \_))
\end{BVerbatim}

\end{document}

