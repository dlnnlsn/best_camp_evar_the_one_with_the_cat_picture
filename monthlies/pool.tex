\documentclass[12pt]{article}
\usepackage{amsmath,amsfonts}
\usepackage[cm]{fullpage}
\usepackage{fancyvrb}

\title{Monthlies Pool}
\author{Stellenbosch Camp 2017}


\begin{document} \maketitle

\begin{itemize}

\item % Scry 2, Magic the Gathering

Consider a deck of $n$ cards with cards numbered from $1$ to $n$. There is one card of each number, but the deck is shuffled, so the cards do not necessarily appear in numerical order. You are allowed to perform the following operation on the deck: You look at the top two cards of the deck, and place one of these two cards on the bottom of the deck, and the other card on top. Show that it possible, using only this operation some number of times, to sort the deck so that the cards appear in ascending order. 


\item % Argentine National Olympiad 2016 1st Level Q2 -- C, Q4
Given 100 infinitely large boxes with finitely many markers in each of them, the following procedure is carried out: At step 1 one adds one marker in every box. At step 2 one marker is added in every box containing an even number of markers. At step 3 one marker is added in every box in which the number of markers is divisible by 3, and so on. Before the process starts Bruno wants to distribute several markers in the boxes such that there is at least one marker in each box and the following holds: after any number of steps there exist two boxes containing a different number of markers. Is it possible for Bruno to do this?


\end{itemize}

\vfill

\centering
\begin{BVerbatim}
\end{BVerbatim}

\end{document}
