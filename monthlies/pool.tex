\documentclass[12pt]{article}
\usepackage{amsmath,amsfonts}
\usepackage[cm]{fullpage}
\usepackage{fancyvrb}

\title{Monthlies Pool}
\author{Stellenbosch Camp 2017}


\begin{document} \maketitle

\begin{itemize}

\item % P Soberon, Problem-Solving Methods in Combinatorics (ladybird --> cat)

On a 2017x2017 board, some of the squares are occupied by a single cat; the rest of the squares are empty. The cats move, never leaving the board, according to the following rules: every second, each cat moves to a neighbouring square, either horizontally (to the square to its immediate left or right) or vertically (to the square below or above its current square); a cat which makes a horizontal move must move vertically on its next move, and a cat which makes a vertical move must move horizontally on its next move. Determine the smallest number of cats such that regardless of their initial positions and their chosen paths, we may ensure that two of them will eventually find themselves in the same square, at the same moment.

\item % 
 

\item % 


\item % 


\end{itemize}

\vfill

\centering
\begin{BVerbatim}
\end{BVerbatim}

\end{document}