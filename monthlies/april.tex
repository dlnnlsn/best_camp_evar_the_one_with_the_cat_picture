\documentclass[12pt]{article}
\usepackage{amsmath,amsfonts}
\usepackage[cm]{fullpage}
\usepackage{graphicx}
\usepackage{enumitem}

\pagenumbering{gobble}

\title{April Monthly Problem Set}
\author{Due: 30 April 2018}
\date{}


\begin{document} \vspace{-12pt} \maketitle \pagestyle{empty}

\begin{enumerate}

\item % DB-2012-11
For integers $x$ and $y$, prove that the last two digits of $31x+73y$ are 18 if and only if the last two digits of $33x+39y$ are 74.


\item % PP-2005-6
$ABCD$ is a rectangle, with points $X$ and $Y$ on sides $AB$ and $BC$ respectively. If the areas of triangles $AXD$, $XBY$ and $YCD$ are $a$, $b$ and $c$, respectively, determine the area of triangle $DXY$ in terms of $a$, $b$ and $c$.


\item % DB-2012-6
In how many ways can you place 16 rooks on a chessboard such that each row contains exactly 2 rooks and each column contains exactly 2 rooks?


\item % Liam & Dylan
Let $(F_n)_{n=1}^\infty$ be a sequence of real numbers such that $F_1 = 1$, $F_2 = 3$ and \[F_{n+1} = \sqrt{F_{n}F_{n+2} +(-1)^{n+1} \cdot 5} \qquad \textrm{for all}\quad n\in\mathbb{N}.\] Prove that $F_n$ is an integer for all $n\in\mathbb{N}$.


\item % JM-2011-1
Show that $\displaystyle \frac{a}{b} +\frac{b}{a} +\frac{c}{d} +\frac{d}{c}$ assumes infinitely many integer values for positive integers $a,b,c,d$ such that $a$ and $b$ have no common divisors and $c$ and $d$ have no common divisors.


\item % SW-2013-6
Let a set $S = \{x_1,x_2,\dotsc,x_n\}$ of $n \geq 4$ real numbers be given. A subset $T = \{x_i,x_j,x_k,x_l\}$ of four distinct elements is called \emph{strange} if the sum of the smallest and the largest elements in $T$ is equal to the sum of the other two. What is the maximum number of strange subsets of $S$?


\item % April Camp 1997 Test 5 Q3
Let $\mathbb{Q}[x]$ denote the set of all polynomials with rational coefficients. Let $f(x) \in \mathbb{Q}[x]$ and let $h(x) = x^3-3x+1$. Suppose $\alpha \in \mathbb{R}$ and $h(\alpha) =h(f(\alpha)) =0$.
\begin{enumerate}[label=(\alph*)]
\item Show that $h(x)$ cannot be written as the product of two nonconstant polynomials in $\mathbb{Q}[x]$.
\item Show that $\alpha$ is not a root of any quadratic polynomial in $\mathbb{Q}[x]$.
\item Hence, or otherwise, show that $h(f^n(\alpha)) = 0$ for all $n \in \mathbb{N}$.
\end{enumerate}


\item % DB&JM-2013-1
Let $ABC$ be a triangle with incentre $I$, and let the incircle and side $BC$ intersect at $D$. Prove that $\sin(\angle IAD) < 1/3$.

\end{enumerate}


\small
\subsection*{Email submission guidelines}

\begin{itemize}
\item Submit each question in a single separate PDF file (with multiple pages if necessary), with your name and the question number written on each page.
\item If you take photographs of your work, use a document scanner such as \verb!CamScanner! to convert to PDF.
\item If you have multiple PDF files for a question, combine them using software such as \verb!PDFsam!.
\end{itemize}

\end{document}
