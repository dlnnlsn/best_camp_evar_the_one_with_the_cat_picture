\documentclass[11pt]{article}
\usepackage{amsmath,amsfonts}
\usepackage[cm]{fullpage}
\usepackage{graphicx}
\usepackage{enumitem}

\pagenumbering{gobble}

\title{June Monthly Problem Set}
\author{Due: 30 May 2018}
\date{\vspace{-24pt}}


\begin{document} \vspace{-12pt} \maketitle \pagestyle{empty}

\begin{enumerate}

\item % Hong Kong TST 3 2017 Q1
\begin{enumerate}
  \item Are there 5 circles in the plane such that each circle passes through exactly 3 centers of other circles?
  \item Are there 6 circles in the plane such that each circle passes through exactly 3 centers of other circles?
\end{enumerate}


\item % Hong Kong TST 3 2017 Q2
Suppose all of the 2018 integers lying in between (and including) 1 and 2018 are written on a blackboard. Suppose we choose exactly 1009 of these numbers and circle each one of them. By the score of such a choice, we mean the square of the difference between the sum of the circled numbers and the sum of the non-circled numbers. What is the average scores over all possible choices for 1009 numbers?


\item % Hong Kong TST 3 2017 Q2
At a mathematical competition $n$ students work on 6 problems each one with three possible answers. After the competition, the Jury found that for every two students the number of the problems, for which these students have the same answers, is 0 or 2. Find the maximum possible value of $n$.


\item % Hong Kong TST 4 2017 Q1
Is there is a permutation $a_1,a_2,\cdots,a_{6666}$ of the numbers $1,2,\cdots,6666$ with the property that the sum $k+a_k$ is a perfect square for all $k=1,2,\cdots,6666$?


\item % Hong Kong TST 4 2017 Q2
Two circles $\omega_1$ and $\omega_2$, centered at $O_1$ and $O_2$, respectively, meet at points $A$ and $B$. A line through $B$ intersects $\omega_1$ again at $C$ and $\omega_2$ again at $D$. The tangents to $\omega_1$ and $\omega_2$ at $C$ and $D$, respectively, meet at $E$, and the line $AE$ intersects the circle $\omega$ through $AO_1O_2$ at $F$. Prove that the length of segment $EF$ is equal to the diameter of $\omega$.


\item % Hong Kong TST 4 2017 Q3
Let a sequence of real numbers $a_0, a_1,a_2, \cdots$ satisfy the following condition for all sufficiently large $m\in\mathbb{N}$: $$\sum_{n=0}^ma_n\cdot(-1)^n\cdot{m\choose n} = 0.$$ Show that there exists a polynomial $P$ such that $a_n = P(n)$ for all $n\geq 0$.


\item % Hong Kong TST 2 2017 Q4
Let $n$ be a positive integer with the following property: $2^n-1$ divides a number of the form $m^2+81$, where $m$ is a positive integer. Find all possible values of $n$.


\item % Hong Kong TST 2 2018 Q4
In triangle $ABC$ with incentre $I$, let $M_A,M_B$ and $M_C$ by the midpoints of $BC, CA$ and $AB$ respectively, and $H_A,H_B$ and $H_C$ be the feet of the altitudes from $A,B$ and $C$ to the respective sides. Denote by $\ell_b$ the line being tangent to the circumcircle of triangle $ABC$ and passing through $B$, and denote by $\ell_b'$ the reflection of $\ell_b$ in $BI$. Let $P_B$ by the intersection of $M_AM_C$ and $\ell_b$, and let $Q_B$ be the intersection of $H_AH_C$ and $\ell_b'$. Defined $\ell_c,\ell_c',P_C,Q_C$ analogously. If $R$ is the intersection of $P_BQ_B$ and $P_CQ_C$, prove that $RB = RC$.

\end{enumerate}


\small
\subsection*{Email submission guidelines}

\begin{itemize}
\item Submit each question in a single separate PDF file (with multiple pages if necessary), with your name and the question number written on each page.
\item If you take photographs of your work, use a document scanner such as \verb!CamScanner! to convert to PDF.
\item If you have multiple PDF files for a question, combine them using software such as \verb!PDFsam!.
\end{itemize}

\end{document}
