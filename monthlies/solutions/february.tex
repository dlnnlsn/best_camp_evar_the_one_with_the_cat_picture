\documentclass[12pt]{article}
\usepackage{amsmath,amsfonts}
\usepackage[cm]{fullpage}
\usepackage{graphicx}

\pagenumbering{gobble}

\title{\vspace{-24pt}February Monthly Problem Set}
\author{Due: 12 February 2018}
\date{}


\begin{document} \maketitle \pagestyle{empty}

\begin{enumerate}

\item % Hellenic Math Olympiad 2017 Juniors Q3
% Determine all positive integers $a,b,p$, where $p$ is prime, satisfying the following equation:
% \[ \frac{1}{p} = \frac{1}{a^2} + \frac{1}{b^2}. \]
The given equation is equivalent to
\[
    p(a^2 + b^2) = a^2 b^2.
\]

We note that $p \mid a^2 b^2$, and so either $p \mid a$ or $p \mid
b$. Without loss of generality, let $p \mid a$, so that $a = pk$ for some
natural number $k$. The equation then becomes
\[
    a^2 + b^2 = pk^2 b^2.
\]

We thus have that $b^2 \mid a^2 + b^2$, and so $b^2 \mid a^2$, which
gives us that $b \mid a$. Let $a = mb$ for some natural number $m$. The
equation then becomes
\[
    p(m^2 + 1) b^2 = m^2 b^4,
\]
or equivalently
\[
    p(m^2 + 1) = m^2 b^2.
\]

Since $\gcd(m^2, m^2 + 1) = 1$, this implies that $m^2 \mid p$, and so $m =
1$. This gives us that $a = b$, and so the equation simplifies to
\[
    2p = b^2,
\]
which implies that $a = b = p = 2$. This is indeed a solution since
\[
    \frac{1}{2} = \frac{1}{4} + \frac{1}{4}.
\]

\item % Easy-Med? Germany 2015
%Let $M$ be the midpoint of side $AB$ of $\triangle ABC$ in which $BC>CA$. The perpendicular bisector of $AB$ meets $BC$ in $P$ and $AC$ extended in $Q$. Let $R$ be the foot of the perpendicular from $P$ onto $AC$ and let $S$ be the foot of the perpendicular from $Q$ onto $BC$. Prove that $M$, $R$ and $S$ lie on a straight line.
Since $\angle PRQ = \angle PSQ = 90^\circ$, we have that $P$, $R$, $S$, and $Q$
lie on the circle with diameter $PQ$. Let $O$ be the midpoint of $PQ$, so that
$O$ is the centre of this circle. Since $O$ lies on the perpendicular bisector
of $AB$, we have that $OA = OB$, and so $A$ and $B$ have equal power with
respect to this circle. It follows that $BP \times BS = AR \times AQ$. By
applying Power of a Point in this circle to point $C$, we also have that $CQ
\times CR = CP \times CS$.

By Menelaus' Theorem applied to triangle $\triangle ABC$ and line $MPQ$, we have that
\[
    \frac{AM}{MB} \cdot \frac{BP}{PC} \cdot \frac{CQ}{CA} = 1.
\]

Using that
\[
    \frac{BP}{QA} = \frac{RA}{BS}
\]
and
\[
    \frac{CQ}{PC} = \frac{SC}{CR},
\]
we thus have that
\[
    \frac{AM}{MB} \cdot \frac{RA}{BS} \cdot \frac{SC}{CR} = 1.
\]

Since
\[
    \frac{AM}{MB} = 1
\]
this is equivalent to
\[
    \frac{AM}{MB} \cdot \frac{BS}{SC} \cdot \frac{CR}{RA} = 1,
\]
and so by Menelaus' Theorem, we have that $M$, $R$, and $S$ are collinear.

\begin{center}
    \includegraphics[width=0.8\textwidth]{february_q2.png}
\end{center}


\item % Scry 2, Magic the Gathering
%Consider a deck of $n$ cards with cards numbered from $1$ to $n$. There is one card of each number, but the deck is shuffled, so the cards do not necessarily appear in numerical order. You are allowed to perform the following operation on the deck: You look at the top two cards of the deck, and place one of these two cards on the bottom of the deck, and the other card on top. Show that it is possible, using only this operation some number of times, to sort the deck so that the cards appear in ascending order. 
We will show by induction on $k$ that for $0 \leq k \leq n$, it is possible to
use the allowed move to arrange the deck such that the cards from $1$ to $k$ are
in ascending order at the bottom of the deck. This clearly implies the desired
result.

The claim is vacuously true for $k = 0$. Now suppose that we have the cards from
$1$ to $k$ in ascending order at the botton of the deck. Repeatedly draw two
cards, and place either one at the bottom of the deck, until one of the two
cards that we draw is $(k + 1)$. At this point, place $(k + 1)$ at the top of
the deck, and the other card on the bottom. Now repeatedly draw two cards, and
place $(k + 1)$ at the top of the deck and the other card at the bottom until
the card that we place is the bottom is card number $k$. At this point, we have
$(k + 1)$ at the top of the deck, and the cards from $1$ to $k$ in ascending
order at the bottom of the deck. Now draw two cards, and place $(k + 1)$ at the
bottom of the deck and the other card at the top so that we have the cards from
$1$ to $(k + 1)$ at the bottom of the deck in ascending order as desired.


\item % DB-2012-7
%Determine all triples of real numbers $(x,y,z)$ that simultaneously satisfy the equations:
%\begin{align*}
%	(x^2 - 4x + 7 )(y^2 + 6y + 14) &= 120 \\
%	(y^2 - 6y + 14)(z^2 + 8z + 23) &= 336 \\
%	(z^2 - 8z + 23)(x^2 + 4x + 7 ) &= 96.
%\end{align*}


\item % Hard? Dutch IMO Selection 2016
%Determine the number of sets $A = \{a_1,a_2,\ldots, a_{1000}\}$ of positive integers satisfying $a_1 < a_2 < \dotsb < a_{1000} \le 2014$ for which the set
%	\[ S = \{a_i+a_j \mid 1\le i,j\le 1000,\ i+j\in A\} \]
%is a subset of $A$.


\item % SW-2011-4
%Determine the largest integer $C$ such that the inequality \[ (x_1^2+200) +(x_2^2+200) +\dotsb +(x_k^2+200) \geq C \] holds for all positive integers $k$ and all $k$-tuples $(x_1,x_2,\dotsc,x_k)$ of positive integers such that $x_1+x_2+\dotsb+x_k = 100$.


\item % JM-2012-4
%Let $ABC$ be a triangle with circumcentre $O$. Let $\omega_1$ and $\omega_2$ be two circles both with centre $O$ but with different sizes. Let $\omega_1$ intersect $AB$ in $P$ and $Q$ (with $P$ closer to $A$) and $\omega_2$ intersect $AC$ in $R$ and $S$ (with $R$ closer to $A$). Let $RQ$ intersect $BC$ at $K$ and let $PS$ intersect $BC$ at $L$. Prove that $BK = CL$.
We make use of directed line segments.

Since $A$ and $B$ are an equal distance from $O$, they have equal power with
respect to $\omega_1$, and so we have that $AP \times AQ = BP \times BQ$.
Simllarly, since $A$ and $C$ are an equal distance from $O$, they have equal
power with respect to $\omega_2$, and so we have that $AR \times AS = CS \times
CR$.

Applying Menelaus' Theorem to triangle $\triangle ABC$ and line $RQK$, we have
that
\[
    \frac{AQ}{QB} \cdot \frac{BK}{KC} \cdot \frac{CR}{RA} = -1.
\]
Applying Menelaus' Theorem to triangle $\triangle ABC$ and line $PSL$ gives us
that
\[
    \frac{AP}{PB} \cdot \frac{BL}{LC} \cdot \frac{CS}{SA} = -1.
\]
Multiplying these together gives us that
\[
    \frac{AP \times AQ}{PB \times QB} \cdot \frac{BL \times BK}{LC \times KC}
    \cdot \frac{CS \times CR}{SA \times RA} = 1.
\]

But we know that
\[
    \frac{AP \times AQ}{PB \times QB} = \frac{CS \times CR}{SA \times RA} = 1.
\]
and so we have that
\[
    BL \times BK = LC \times KC.
\]

We thus have that
\[
    BK (BC + CL) = LC (KB + BC)
\]
and so
\[
    BK \times BC + BK \times CL = BK \times CL + LC \times BC
\]
giving us that
\[
    BK = LC
\]
as desired.

\begin{center}
    \includegraphics[width=0.8\textwidth]{february_q7.png}
\end{center}

\item % LB-2010-2
%Find all functions $f : \mathbb{Z} \to \mathbb{Z}$ such that for each pair of integers $a$ and $b$, where $a < b$, there exists a prime number $p$ and a nonnegative integer $c$ such that \[ \frac{f(b)-f(a)}{b-a} = p^c. \]


\end{enumerate}

\end{document}
