\documentclass[12pt]{article}
\usepackage{amsmath,amsfonts}
\usepackage[cm]{fullpage}
\usepackage{fancyvrb}

\title{January Monthly Problem Set}
\author{Due: 15 January 2018}
\date{}

\begin{document} \maketitle

\begin{enumerate}

\item % JMMO 2017 Q1
  Let $p$ be a prime number such that $3p+10$ is equal to the sum of the squares of six consecutive positive integers. Prove that $36 \mid p-7$.

 
\item % Swiss 2017 Q1
Let $A$ and $B$ be points on a circle $k$ with centre $O$ such that $AB > AO$. Let $C \neq A$ be the second intersection point of the angle bisector of $\angle OAB$ with $k$. Let $D \neq B$ be the intersection of the line $AB$ with the circumcircle of $OBC$. Prove that $AD = AO$.


\item % Croatia
Find all functions $f : \mathbb{R} \to \mathbb{R}$ such that
  \[ xf(x) - yf(y) = (x - y)(f(x + y) - xy) \]
for all real numbers $x$ and $y$.


\item % P Soberon, Problem-Solving Methods in Combinatorics (ladybird --> cat)
On a 2017x2017 board, some of the squares are occupied by a single cat; the rest of the squares are empty. The cats move, never leaving the board, according to the following rules: every second, each cat moves to a neighbouring square, either horizontally (to the square to its immediate left or right) or vertically (to the square below or above its current square); a cat which makes a horizontal move must move vertically on its next move, and a cat which makes a vertical move must move horizontally on its next move. Determine the smallest number of cats such that regardless of their initial positions and their chosen paths, we may ensure that two of them will eventually find themselves in the same square, at the same moment.


\item % German NMO 2016 Second Round Q4
Each side of a regular dodecahedron detmines a plane in a unique way. These planes divide the space into a finite number of distinct parts. Determine the number of these parts.

\small{Remark: These planes or parts of them are not considered as parts of the space.}


\item % Dutch TST 2016
Find all functions $f: \mathbb{R} \to \mathbb{R}$ satisfying
  \[ f(xy-1) + f(x)f(y) = 2xy-1 \]
for all $x,y\in R$.


\item % Hard Dutch IMO Selection 2016
Let $\Gamma_1$ be a circle with centre $A$ and $\Gamma_2$ be a circle with centre $B$ such that $A$ lies on $\Gamma_2$. On $\Gamma_2$ there is a (variable) point $P$ not lying on $AB$. A line through $P$ is a tangent of $\Gamma_1$ at $S$, and it interesects $\Gamma_2$ again in $Q$, with $Q$ and $P$ lying on the same side of $AB$. A different line through $Q$ is tangent to $\Gamma_1$ at $T$. Moreover, let $M$ be the foot of the perpendicular to $AB$ through $P$. Let $N$ be the intersection of $AQ$ and $MT$.

Prove that $N$ lies on a line independent of the position of $P$ on $\Gamma_2$.

\item % Med-Hard? Dutch IMO Selection 2016
Let $k$ be a positive integer, and let $s(n)$ denote the sum of the digits of $n$. Show that among all positive integers with $k$ digits, there are as many numbers $n$ satisfying $s(n) < s(2n)$ as there are numbers $n$ satisfying $s(n) > s(2n)$.

\end{enumerate}

\vfill

\centering
\begin{BVerbatim}
\end{BVerbatim}

\end{document}
